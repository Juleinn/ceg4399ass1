\documentclass[a4paper,man,natbib,12]{apa6}

\usepackage[english]{babel}
\usepackage[utf8x]{inputenc}
\usepackage{amsmath}
\usepackage{graphicx}
\usepackage[colorinlistoftodos]{todonotes}
\usepackage{apacite}


\linespread{2}


\title{Assignment \# 1}
\shorttitle{}
\author{Anton CLAES}
\affiliation{University of Ottawa}

\begin{document}
\thispagestyle{empty}

\begin{center}
\vspace*{5\baselineskip}
Assignment \#1

Design of Secure Computer Systems

\vspace*{10\baselineskip}

Anton Claes

0300042110

CEG4399

University of Ottawa

September 25th, 2017

\newpage

Assignment \# 1
\end{center}

\qquad One of the most impressive computer system attacks performed recently is the attack against the Dyn company. Dyn's main business is Domain Name Servers managment. The company hosts canonical website names in it's DNS servers and provides the corresponding IP addresses to the users performing DNS lookups on these sites. The attack against the Dyn company occured on October 21st, 2016 and targetted Dyn's DNS servers. 

\qquad The attack is a Distributed Denial of Service (DDoS) Attack : "A Distributed Denial of
Service (DDoS) attack is a coordinated attack on the
availability of services of a given target system or network
that is launched indirectly through many compromised
computing systems", \cite{DDoS}. Although DNS servers are usualy designed to handle huge amounts of requests, they can be taken down with enough devices attacking at the same time. 

\qquad What's interesting with this kind of attack is that the breach is not inside the attacked system itself, but in the compromised devices attacking it. According to Dyn's report, the attack was caused by a botnet called Mirai, which performed 2 attacks on that day (One in the morning, the second one in the afternoon)(\cite{Dyn}). The Mirai botnet is a malicious piece of software that targets Internet of Things (IoT) devices and connected objects. Mirai uses the fact that a lot of these devices are poorly secured and left with default passwords (\cite{46301}, p. 7). Thus, to infect other devices and spread accross the network, Mirai has a table storing default passwords for some devices. It then scans the network and once it has found a host, it tries to log into it with one of the passwords stored in the table. It is then able to replicate itself in the device. (\cite{46301}, p. 2). 

\qquad Once the botnet is widely spread accross the network, all the devices perform DNS queries on the Dyn's servers at the same time, thus generating huge bandwith : "Early observations of the TCP attack volume from a few of our datacenters indicate packet flow bursts 40 to 50 times higher than normal", \cite{Dyn}. 

\qquad This attack caused major disruptions accross the internet : "The incident took offline some of the most popular sites on the web, including Netflix, Twitter, Spotify, Reddit, CNN, PayPal, Pinterest and Fox News", \cite{guardian}. Although the websites being inaccessible do not cause data loss or privacy issues, it obviously represents a huge money loss, especially for companies like Spotify which only source of revenue is through the internet. For the Dyn company, there weren't any long-term consequences : "we were able to substantially recover from the second attack by 17:00 UTC", \cite{Dyn}.

\qquad Interesting fact about this attack, it is not actually due to a design error, but it's been made possible by people not changing their default passwords. Altough it's arguable that compromised devices manufacturers could have designed their products to force password change on install, or delivered all devices with different random passwords, they can not really be blamed for not having done it. Users cannot be blamed neither, because changing passwords is wise, but not mandatory. By essence, these kind of attacks are hard to avoid : "During a DDoS which uses the DNS protocol it can be difficult to distinguish legitimate traffic from attack traffic", \cite{Dyn}. 

\qquad There are, though, attempts in preventing the attacks. This can for instance be achieved with specific firewalls. XFirewall is one of them, it is capable of detecting and dynamically filtering the traffic in case of an attack : "XFirewall is configured with dynamic rules based on traffic analysis. Once the DDoS storm is over, XFirewall is removed.", \cite{XFirewall}, p1-2. Another solution is to monitor the traffic closer to the attacker's side. Another solution that has been proposed is to have the system distributed to make the attack harder : "It is evident that no single deployment point can achieve successful defense in autonomous operation. Therefore, the
DDoS problem requires a distributed solution in which defense nodes located throughout the Internet cooperate to achieve better overall defense", \cite{Mirkovic:2003:AFD:986655.986658}, p. 13. There are numerous solutions that have been proposed against DDoS attacks over the years, but the DDoS attack on Dyn's servers occured anyways. According to \cite{Mirkovic:2003:AFD:986655.986658}, all these solutions are too specific and attackers will find a way around (p. 1).

\qquad To conclude, this attack, by directly targetting the Dyn company, also indirectly attacked all the companies which canonical names were hosted by Dyn. Considering how many big companies were impacted, and even if the attack lasted only one day, it can be considered one of the major attacks that occured on the internet these days. By using the weakness of poorly secured devices, it was abble to compromise a huge part of the web for one day. Although solutions exist against these attacks, they can be hard to implement and they may not always be efficient. The main issue to be addressed in the first place is probably the compromised IoT devices, because it was millions of them, which can not be considered a small issue. As this attack has proven, the DDoS attacks can occur and target huge systems. Solutions exist, but they may not be as efficient as needed, so only reliable solution may be to re-think the whole system with the idea of making it resistant to these attacks.

\bibliography{assignment1.bib}


\end{document}
